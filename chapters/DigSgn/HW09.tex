%!TeX root = ../../ccc_main.tex
%!TeX spellcheck = en_GB

\subsection{Hohenberger-Waters RSA Short Signature}
The first efficient RSA signature produced without relying on tree structures was presented in \cite{C:HohWat09}. Their construction make use of the generic transformation from weakly-secure signatures and a chameleon hash function \EG{refs, especially for the RSA chameleon hash}.
In order to produce a weak signature they introduce the \textit{prefix-guessing} technique, which allows in the security reduction to sign required messages while still being able to retrieve non-trivial information from a forge.

At a high level for any message $m \in \{0, 1\}^n$ let $m^{(i)}$ be its prefix of length $i$, i.e. its first $i$ bits. 
If the adversary requires signatures of $m_1, \ldots, m_Q$ and forges a messages $\hat{m}$, then there exists at least a message $m_{i^\ast}$ with the longest common prefix with $\hat{m}$.
In other words $\hat{m}^{(t^\ast)} = m_{i^\ast}^{(t^\ast)}$.
The reduction can then correctly guess $i^\ast$ and $t^\ast$ with significant probability and ``embed'' the challenge in all those message whose $t^\ast$ first bits agree with $m_{i^\ast}$ and the $t^\ast + 1$-th don't, while still being able to simulate the other messages.
Notice that the $m_i$ don't satisfy this requirement, and that knowing them in advance (even before producing public parameters) is essential to correctly produce this partition.

Finally they apply this technique using a programmable target collision-resistant hash function $H : \{0, 1\}^\ast \to \{0, 1\}^\ell$ mapping strings to prime numbers.
This can be obtained from a PRF. 
A key is then a PRF's key $k$ and $c \rgets \{0, 1\}^\ell$ and $H(m) \coloneqq f_k(j, m) \oplus c$ where $j$ is the smallest integer that makes $f_k(j, m) \oplus c$ a prime number.

\begin{figure}[htb]
\centering
\begin{pcvstack}
	\algorithm{
		$\cd{Setup}:$
		}
		{
			Sample $H$ programmable TCR hash function
				\\
			$p_1, p_2 \rgets \range{2^\ell}$ primes, $\;$ $N \gets p_1 \cdot p_2$, $\;$ $h \rgets \Z_N^\ast$
				\\
			$\vk \gets (H, N, h), \; \sk \gets \phi(N)$. Return $\vk, \sk$
		}
	\algorithm{
		$\cd{Sign}(m, \sk):$
		}
		{
			Set $m^{(i)}$ the first $i$ bits of $m \in \{0, 1\}^n$
				\\
			$e_i \gets H(m^{(i)}), \quad \sigma \gets h^{e_1^{-1} \cdot \ldots \cdot e_n^{-1}}$
				\\
			Return $\sigma$
		}
	\algorithm{
		$\cd{Vfy}(m, \sigma, \cd{vk}):$
		}
		{
			Set $m^{(i)}$ the first $i$ bits of $m \in \{0, 1\}^n$
				\\
			$e_i \gets H(m^{(i)}). \;$ Return $h \checkeq \sigma^{e_1 \cdot \ldots \cdot e_n}$
		}
\end{pcvstack}
\label{prot:HohWat09}
\caption{Short RSA signature \cite{C:HohWat09}. $H$ maps to primes of length $\ell$}
\end{figure}

\begin{proposition}
	\label{prop:HohWat09}
	The scheme detailed in Figure \ref{prot:HohWat09} is a weakly secure signature if the RSA problem is hard.
\end{proposition}
