%!TeX root = ../../ccc_main.tex
%!TeX spellcheck = en_GB

\subsection{From Claw-Free Permutation}

The first general construction of chameleon hash functions was provided in \cite{NDSS:KraRab00} based on claw-free trapdoor permutations \EG{ref}. 
First of all they observe that any chameleon hash with small message space can be extended over the set of all strings by composing with a collision-resistant hash function.
This clearly preserves collision resistance and, given a trapdoor for the underlying smaller chameleon hash, for any $y$ in the output space and $m \in \{0, 1\}^\ast$ one can find the right $r$ such that $(h(m), r)$ is mapped to $y$.

Next, they provide a chameleon hash applying to a random element $r$ subsequently either $f_0$ or $f_1$ according to the $i$-th bit of $m$, where $(f_0, f_1)$ is the pair of trapdoor permutations.
If an adversary manages to find a collision this directly translate into a claw for $f_0, f_1$.

\begin{figure}[htb]
\centering
\begin{pcarray}{l}
	\algorithm{
		$\cd{Setup}:$
		}
		{
			Sample permutations $f_0, f_1$ and trapdoors $f_0^{-1}, f_1^{-1}$
				\\
			$\cd{key} \gets (f_0, f_1), \; \cd{td} \gets (f_0^{-1}, f_1^{-1})$.
			Return $(\cd{key}, \cd{td})$
		}
		\\
	\algorithm{
		$\cd{Hash}(m, r, \cd{key}):$
		}
		{
			Parse $m = (b_1, \ldots, b_n)$ with $b_i \in \{0, 1\}$
				\\
			$y \gets f_{b_n} \circ \ldots \circ f_{b_1} (r)$. Return $y$
		}
		\\
	\algorithm{
		$\cd{Forge}(y, m, \cd{td}):$
		}
		{
			Parse $m = (b_1, \ldots, b_n)$ with $b_i \in \{0, 1\}$
				\\
			$r \gets f_{b_1}^{-1} \circ \ldots \circ f_{b_n}^{-1}(y)$. Return $r$
		}
\end{pcarray}
\caption{Chameleon hash from claw-free permutations in \cite{NDSS:KraRab00}, with $m \in \{0, 1\}^n$}
\label{prot:KraRab:trapdoor_permutations}
\end{figure}

\begin{proposition}
	If $f_0, f_1$ is a pair of claw-free trapdoor permutations, then the construction in figure \ref{prot:KraRab:trapdoor_permutations} is a chameleon hash
\end{proposition}


\subsection{From Factoring}
In \cite{NDSS:KraRab00} a concrete chameleon hash is provided based on the hardness of factoring by roughly instantiating the generic construction based on claw-free trapdoor permutations.
To do so they simply provide such a primitive whose claw-freeness is implied by factoring.
This is done by setting, for a given RSA modulus $N = pq$ with $p = 3 \mod 8$ and $p = 7 \mod 8$ by setting
\[	
	f_0 : \QR \to \QR
		\: : \:
	f_0(x) = x^2,
		\qquad
	f_1 : \QR \to \QR
		\: : \:
	f_1(x) = 4 x^2.
\]
The restriction imposed on $p, q$ guarantees that $-1$ is not a square both modulo $p$ and $q$, $2$ is not a quadratic residue $\mod p$ but it is $\mod q$.
Notice that knowing the factorisation of $N$ one can invert those functions by computing through chinese remainder theorem a square root that is itself a quadratic residue.
In conclusion one can prove the following

\begin{proposition}
	If factoring is hard then $f_0, f_1$ is a pair of claw-free trapdoor permutations on $\QR$
\end{proposition}

However deciding membership in $\QR$ is hard so an adversary breaking the resulting chameleon hash obtain directly from \ref{prot:KraRab:trapdoor_permutations} may find collision for values of $r \in \Z_N \setminus \QR$ leaving the reduction unable to tell whether this is a valid collision which translate into a claw for $f_0, f_1$.
For this reason one need to adapt the general construction by composing with $g : \Z_N \to \QR$ such that $g(x) = x^2$.

\begin{figure}[htb]
\centering
\begin{pcarray}{l}
	\algorithm{
		$\cd{Setup}:$
		}
		{
			$p, q \rgets \{0, 1\}^\ell$ primes s.t. $p \bmod 8 = 3, \; q \bmod 8 = 7$
				\\
			$N \gets pq, \; \cd{key} \gets N, \; \cd{td} \gets (p, q)$. Return $\cd{key}, \cd{td}$
		}
		\\
	\algorithm{
		$\cd{Hash}(m, r, \cd{key}):$
		}
		{
			Parse $m = (b_1, \ldots, b_n)$ with $b_i \in \{0, 1\}$
				\\
			$y \gets f_{b_n} \circ \ldots \circ f_{b_1} (r^2)$. Return $y$
		}
		\\
	\algorithm{
		$\cd{Forge}(y, m, \cd{td}):$
		}
		{
			Parse $m = (b_1, \ldots, b_n)$ with $b_i \in \{0, 1\}$
				\\
			$r \gets f_{b_1}^{-1} \circ \ldots \circ f_{b_n}^{-1}(y)$. Return $r$
		}
\end{pcarray}
\caption{Chameleon hash from factoring in \cite{NDSS:KraRab00}, $m \in \{0, 1\}^n$}
\label{prot:KraRab:factoring}
\end{figure}

\begin{proposition}
	If factoring is hard, the construction in Figure \ref{prot:KraRab:factoring} is a chameleon hash function
\end{proposition}

\subsection{From Discrete Log}
Still in \cite{NDSS:KraRab00} a chameleon hash based on discrete log is proposed. The idea could be generalised to produce chameleon hash from any perfectly hiding equivocal commitment.
This scheme however achieve mildly weaker forging property as $\cd{Forge}$ algorithm requires in input not only an image $y$ but also the message and randomness used to generate it.

\begin{figure}[htb]
\centering
\begin{pcarray}{l}
	\algorithm{
		$\cd{Setup}:$
		}
		{
			$g \rgets \G, \; x \rgets \F_q, \; h \gets g^x$
				\\
			$\cd{key} \gets (g, h), \; \cd{td} \gets x$. Return $\cd{key}, \cd{td}$
		}
		\\
	\algorithm{
		$\cd{Hash}(m, r, \cd{key}):$
		}
		{
			$y \gets g^m h^r$. Return $y$
		}
		\\
	\algorithm{
		$\cd{Forge}((y, m', r'), m, \cd{td}):$
		}
		{
			$r \gets r' + x^{-1}(m' - m)$. Return $r$
		}
\end{pcarray}
\caption{Chameleon hash from Discrete Log in \cite{NDSS:KraRab00}, $m \in \F_q$. Procedure $\cd{Forge}$ requires $m', r'$ such that $y = \cd{Hash}(m', r', \cd{key})$}
\label{prot:KraRab:dlog}
\end{figure}

\begin{proposition}
	If the discrete logarithm problem is hard in $\G$, the construction in Figure \ref{prot:KraRab:dlog} is a chameleon hash function
\end{proposition}








