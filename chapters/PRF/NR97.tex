%!TeX root = ../../ccc_main.tex
%!TeX spellcheck = en_GB

\subsection{Naor-Reingold PRF}
The first PRF based on DDH was proposed by Naor and Reingold in \cite{FOCS:NaoRei97}, building on \cite{FOCS:GolGolMic84}.
The key features of this function family are that it is computable in $\cd{TC}^0$ with preprocessing, as it only involves a subset product, it is algebraic, i.e.\ it can be instantiated in the Generic Group Model, and admits efficient sigma protocols to show the correct evaluation under a committed key of a publicly known input, something used in Bellare-Goldwasser \cite{C:BelGol89} signatures.
Moreover its security proof is \textit{tight}, meaning that is does not depend on the number of queries performed by the adversary.
In the following $f_k: \{0, 1\}^n \to \G$ and $k \in \F_q^{n+1}$ with $q$ being the prime order of $\G$.
In order to have domain $\{0, 1\}^m$ standard techniques using unique group element representation and universal hashing can be applied.

\begin{figure}[htb]
\centering
\begin{pcarray}{ll}
	\algorithm{
		$\cd{Setup}(1^\secp):$
		}
		{
			$(a_0, a_1, \ldots, a_n) \rgets \F_q^{n+1}$
				\\
			Return $(a_0, a_1, \ldots, a_n)$
		}
			&
	\algorithm{
		$f_k(x):$
		}
		{
			Parse $x = (b_1, \ldots, b_n)$
				\\
			Return $\gexp{}{a_0 \cdot \prod_{b_i = 1} a_{b_i}}$
		}
\end{pcarray}
\caption{PRF from DDH in \cite{FOCS:NaoRei97} with $f_k: \{0, 1\}^n \to \G$.}
\label{prot:NaoRei88:prf_from_ddh}
\end{figure}

At the core of its tight security proof lies a reduction of $\ddh$ to $\ddh_n$ which does not depend on $n$. Formally for any $\ppt$ $\mathcal{D}$ its advantage in the $\ddh_n$ game is defined as
\[
	\adv{\mathcal{D}}
		\; = \;
	\left|
		\prob{\mathcal{D}(\gexp{}{\vc{u}}, \gexp{}{\vc{v}})}
			-
		\prob{\mathcal{D}(\gexp{}{\vc{u}}, \gexp{}{\alpha \vc{u}})}
	\right|
\]
where $\vc{u}$ and $\vc{v}$ are random vectors in $\F_q^n$.

\begin{lemma}
	For any $\mathcal{A}$ playing against $\ddh_n$ there exists $\mathcal{B}$ playing against $\ddh$ such that $\adv{\mathcal{B}} = \adv{\mathcal{A}}$.
\end{lemma}
\begin{proof}
	Calling $\gexp{}{a}$, $\gexp{}{b}$, $\gexp{}{c}$ the elements received by $\mathcal{B}$, the key idea is that this tuple is Diffie-Hellman if and only if the matrix
	\[
		M 
			\; = \;
		\left(
		\begin{matrix}
			1 & a \\
			b & c 
		\end{matrix}
		\right)
	\]
	has rank $1$. With this idea in mind $\mathcal{B}$ samples $\vc{x}, \vc{y} \rgets \F_q^n$ and computes in the exponent
	\[
		(\vc{u}, \vc{v}) = M \cdot (\vc{x}, \vc{y})
	\]
	where $(\vc{x}, \vc{y})$ is a column vector.
	If $M$ has rank $1$, $\vc{u}, \vc{v}$ will be linearly dependent, and so $\mathcal{B}$ simulated $\ddh[1]_n$. 
	Otherwise $M$ has rank $2$ and in particular $\vc{u}, \vc{v}$ are uniformly and independently distributed vectors, meaning $\mathcal{B}$ perfectly simulated $\ddh[0]_n$.
\end{proof}

\begin{proposition}
	\label{prop:NaoRei97}
	The function defined in Fig.~\ref{prot:NaoRei88:prf_from_ddh} is a PRF under the assumption that $\ddh$ is hard.
	More specifically, for any $\mathcal{A}$ breaking the PRF security, there exists $\mathcal{B}$ breaking $\ddh$ with advantage
	\[
		\adv{\mathcal{B}} = \frac{1}{n} \cdot \adv{\mathcal{A}}.
	\]
\end{proposition}
\begin{proof}[Proof Sketch]
	The proof consists of $n+1$ hybrid game, where in $\cd{G}_0$ $\mathcal{A}$ access $f_k(\emptyentry)$, whereas in $\cd{G}_n$ it access a fully random function.
	Game $\cd{G}_i$ is defined through a random $g: \{0, 1\}^i \to \F_q$. 
	Each query input $x$ is parsed as $x = \vc{b}^\ast || \vc{b}$ with $\vc{b}^\ast \in \{0, 1\}^i$ and $\vc{b} \in \{0, 1\}^{n-i}$ and is replied with
	\[
		\gexp{}{g(\vc{b}) \cdot \prod\nolimits_{b_i = 1} a_i}.
	\]
	Next we show that a $\ppt$ adversary $\mathcal{D}$ distinguishing $\cd{G}_{i-1}$ from $\cd{G}_i$ making $Q$ queries can be used to break $\ddh_{Q}$.
	The key idea is that on input $\gexp{}{\vc{u}}$ and $\gexp{}{\vc{v}}$ will lazily emulate $g$ by either using for the $j$-th query $\gexp{}{u_j}$ or $\gexp{}{v_j}$ as a base group element (the former if $b_i = 0$ and the latter if $b_i = 1$).
	If the same prefix $\vc{b}^\ast$ is queried more than once, the same $\gexp{}{u_j}$, $\gexp{}{v_j}$ pair is used.
	Finally, it raises the chosen base elements correctly according to the remaining bits $b_{i+1}, \ldots, b_n$. 
	A reduction is described in Figure~\ref{prot:NaoRei97:DDH_reduction}
	
	\begin{figure}[htb]
	\centering
	\algorithm{$\mathcal{B}(\gexp{}{\vc{u}}, \gexp{}{\vc{v}})$}{
		Parse $\gexp{}{\vc{u}} = (\gexp{}{u_1}, \ldots, \gexp{}{u_Q})$ and $\gexp{}{\vc{v}} = (\gexp{}{v_1}, \ldots, \gexp{}{v_Q})$
			\\
		Sample $a_{i+1}, \ldots, a_n \rgets \F_q$
			\\
		Maintain an initially empty ordered set $L = \varnothing$
			\\
		Run $\mathcal{A}$
			\\
		\textbf{When} $\mathcal{A}$ queries $x = (\vc{b}^\ast, \vc{b})$:
			\\
		\t Add $\vc{b}^\ast$ in $L$ and get the index $j$ of $\vc{b}^\ast$ in $L$
			\\
		\t \textbf{If} $b_i = 0$:
			\\
		\t\t Send $\gexp{}{u_j \cdot \prod_{b_\ell = 1}^{\ell > i} a_\ell}$ to $\mathcal{D}$
			\\
		\t \textbf{If} $b_i = 0$:
			\\
		\t\t Send $\gexp{}{v_j \cdot \prod_{b_\ell = 1}^{\ell > i} a_\ell}$ to $\mathcal{D}$
			\\
		\textbf{When} $\mathcal{D} \to b$
			\\
		\t Return $b$
	}
	\label{prot:NaoRei97:DDH_reduction}
	\caption{Algorithm $\mathcal{B}$ reducing a distinguisher for $\cd{G}_{i-1}$ and $\cd{G}_i$ to $\ddh_n$}
	\end{figure}
	
	It is easy to see that if $\vc{v}$ is a multiple of $\vc{u}$, $\mathcal{B}$ simulates $\cd{G}_{i-1}$ whereas, if $\vc{v}$ is random it simulates $\cd{G}_i$.
	The thesis follows observing that the hybrid argument is applied $n$ times.
\end{proof}
