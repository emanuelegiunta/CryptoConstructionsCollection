%!TeX root = ../../ccc_main.tex
%!TeX spellcheck = en_GB

\subsection{ABDP Selectively Secure IPFE from DDH}
\EG{I'm using the group additively, this should be done in other examples too since the notation is much much clearer}
One of the first functional encryption scheme was proposed in \cite{PKC:ABDP15}.
The construction, which can be generalized to work with black box encryption schemes with certain homomorphic properties, can be summarized as a vector ElGamal encryption with common randomness.
The construction only uses a prime order group, with no pairings and of known order.
Notice that similarly to other group-based FE scheme, the decryption result is obtained in the exponent, and can then be retrieved by brute-force discrete logarithm algorithms efficiently as long as it lies in a polynomially bounded range.
In terms of security it is only selectively secure in the standard model, however it can be proved adaptively secure in the GGM \EG{not in the original paper, should I prove it?}.

\begin{figure}[htb]
\centering
\begin{pcarray}{lll}
	\algorithm{
		$\cd{Setup}(1^\secp, n):$
		}
		{
			$\vc{v} \rgets \F_q^n$
				\\
			$\mpk \gets \gexp{}{\vc{v}}, \quad \msk \gets \vc{v}$
				\\
			Return $(\mpk, \msk)$
 		}
	&
	\algorithm{
		$\cd{KeyGen}(\vc{d}, \msk):$
		}
		{
			Return $\sk_\vc{d} \gets \vc{d}^\trs \vc{v}$
		}
	\\
	\algorithm{
		$\cd{Enc}(\vc{x}, \mpk):$
		}
		{
			$s \rgets \F_q$
				\\
			$c \gets \left( \gexp{}{s}, 
				\; 
			\gexp{}{\vc{x} + s \vc{v}} \right)$
				\\
			Return $c$
		}
	&
	\algorithm{
		$\cd{Dec}(c, \sk_\vc{d}, \mpk):$
		}
		{
			Parse $c = (c_0, \vc{c}_1)$
				\\
			$m \gets \vc{d}^\trs \vc{c}_1 - \sk_\vc{d} \cdot c_0$
				\\
			Return $m$
		}
\end{pcarray}
\caption{ABDP Selectively Secure Inner Product Functional Encryption.}
\label{prot:ABDP15:IPFE}
\end{figure}

The scheme is correct since, using the same notation of Fig.~\ref{prot:ABDP15:IPFE}
\[
	m
		\; = \;
	\gexp{}{\vc{d}^\trs \vc{x} + s \cdot \vc{d}^\trs \vc{x}}
		-
	\gexp{}{s \cdot \vc{d}^\trs \vc{x}}
		\; = \;
	\gexp{}{\vc{d}^\trs \vc{x}}.
\]

\begin{proposition}
	The scheme in Fig.~\ref{prot:ABDP15:IPFE} is a Selectively Secure Functional Encryption with Inner Product functionality under the $\ddh$ assumption. 
\end{proposition}
\begin{proof}[sketch]
	We provide a direct reduction $\mathcal{B}$ to $\ddh$ given an adversary $\mathcal{A}$ for the indistinguishability game.
	On input $\gexp{}{\alpha}, \gexp{}{\beta}, \gexp{}{\gamma}$, $\mathcal{B}$ executes the adversary which will query $\vc{x}_0, \vc{x}_1$. Then it samples a random bit $b$ and compute the master public key and challenge ciphertext as
	\[
		\mpk
			\; = \;
		\gexp{}{\vc{u} + \alpha(\vc{x}_1 - \vc{x}_0)}
			\qquad
		c
			\; = \;
		\left( \gexp{}{\beta}, \; \gexp{}{\vc{x}_b + \beta \vc{u} + \gamma(\vc{x}_1 - \vc{x}_0)} \right)
	\]
	with $\vc{u}$ uniformly sampled over the linear space $V \subseteq \F_q^n$ orthogonal to $\vc{x}_1 - \vc{x}_0$.
	When the adversary queries the secret key for a vector $\vc{d}$ the reduction replies with
	\[
		\sk_\vc{d}
			\; = \;
		\vc{d}^\trs \vc{u}.
	\]
	Finally when the adversary guesses a bit $b'$ the reduction returns $1$ if the guess was correct, i.e. if $b' = b$, $0$ otherwise.
	
	Regardless of which experiment $\mathcal{B}$ is executed in, secret keys are generated correctly if $\mathcal{A}$ only performs legitimate queries, that are vectors $\vc{d}$ such that $\vc{d}^\trs \vc{x}_0 = \vc{d}^\trs \vc{x}_1$ as this implies $\vc{d}^\trs(\vc{x}_1 - \vc{x}_0)$ and in particular
	\[
		\sk_\vc{d}	
			\; = \;
		\vc{d}^\trs \vc{v}
			\; = \;
		\vc{d}^\trs \left( \vc{u} + \alpha(\vc{x}_1 - \vc{x}_0) \right)
			\; = \;
		\vc{d}^\trs \vc{u}.
	\]
	If $\mathcal{B}$ is executed in $\ddh[1]$, that is with $\gamma = \alpha \beta$, the challenge ciphertext is also correctly generated as the exponent of the second component becomes $\beta$ times the exponent of the master public key.
	Conversely if $\mathcal{B}$ is executed in $\ddh[0]$, $\gamma$ is uniformly random and in particular $\vc{x}_b + \gamma(\vc{x}_0 - \vc{x}_1)$ is uniformly distributed over the affine line containing $\vc{x}_0$ and $\vc{x}_1$.
	Thus $\mathcal{A}$ has no information on $b$ and cannot guess better than at random.
	In conclusion
	\begin{align*}
		\adv{\mathcal{B}}
			\; &= \;
		\left| \prob{\mathcal{A} \text{ wins} \,|\, \ddh[1]} - \prob{\mathcal{A} \text{ wins} \,|\, \ddh[0]} \right|
			\\ &= \;
		\left| \frac{1}{2} + \frac{1}{2} \cdot \adv{\mathcal{A}} - \frac{1}{2} \right|
			\; = \;
		\frac{1}{2} \cdot \adv{\mathcal{A}}.
	\end{align*}
\end{proof}