%!TeX root = ../../ccc_main.tex
%!TeX spellcheck = en_GB

\subsection{Waters IBE}

The first construction in the standard model using constant assumptions is presented in \cite{EC:Waters05}. 
For simplicity, we present here the original version using symmetric pairings - even though it could be adapted to asymmetric ones.
Historically, this is one of the first constructions that successfully used a partitioning technique in order to simulate the secret key required, but at the same time embed the underlying problem in the challenge ciphertext.

\begin{figure}[htb]
\centering
\begin{pcarray}{lll}
	\algorithm{
		$\cd{Setup}:$
		}
		{
			$\alpha, \beta \rgets \F_q, \quad \vc{w} \rgets \F_q^{n+1}$
				\\
			$\mpk \gets (\gexp{}{\alpha}, \gexp{}{\beta}, \gexp{}{\vc{w}}),
				\quad
			\msk \gets \gexp{}{\alpha \beta}$
				\\
			Return $(\mpk, \msk)$
		}
	&
	\algorithm{
		$\cd{KeyGen}(\vc{d}, \msk):$
		}
		{
			$r \rgets \F_q$
				\\
			$\sk_\vc{d} \gets (\gexp{}{r}, \gexp{}{\alpha \beta + r \vc{w}^\trs \vc{d}})$
				\\
			Return $\sk_\vc{d}$
		}
	\\
	\algorithm{
			$\cd{Enc}(m, \vc{d}, \mpk):$
			}
			{
				$\rho \rgets \F_q$
					\\
				$c_0 \gets \gexp{}{\rho}
					, \quad 
				c_1 \gets \gexp{1}{\rho \vc{w}^\trs \vc{d}}$
					\\
				$c_2 \gets m \cdot e(\gexp{}{\alpha}, \gexp{}{\beta})^\rho$
					\\
				Return $(c_0, c_1, c_2)$
			}
	&
	\algorithm{
		$\cd{Dec}(c, \sk_\vc{d}, \mpk):$
		}
		{
			Parse $c = (c_0, c_1, c_2)$
				\\
			Parse $\sk_\vc{d} = (\sigma_1, \sigma_2)$
				\\
			$m \gets c_2 \cdot e(c_0, \sigma_2) \cdot e(c_1, \sigma_1)^{-1}$
				\\
			Return $m$
		}
\end{pcarray}
\caption{Waters IBE. Identities are of the form $\vc{d} \in \{1\} \times \{0, 1\}^n$}
\label{prot:Waters05:IBE}
\end{figure}

\begin{proposition}
	The scheme in Fig.~\ref{prot:Waters05:IBE} is an $\cd{IND-CPA}$ IBE under the $\cd{BDH}$ assumption
\end{proposition}