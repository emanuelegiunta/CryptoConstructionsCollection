%!TeX root = ../../ccc_main.tex
%!TeX spellcheck = en_GB

\subsection{Waters IBE}

The first construction in the standard model using constant assumptions is presented in \cite{EC:Waters05}. 
For simplicity, we present here the original version using symmetric pairings - even though it could be adapted to asymmetric ones.
Historically, this is one of the first constructions that successfully used a partitioning technique in order to simulate the secret key required, but at the same time embed the underlying problem in the challenge ciphertext.

\begin{figure}[htb]
\centering
\begin{pcarray}{lll}
	\algorithm{
		$\cd{Setup}:$
		}
		{
			$\alpha, \beta \rgets \F_q, \quad \vc{w} \rgets \F_q^{n+1}$
				\\
			$\mpk \gets (\gexp{}{\alpha}, \gexp{}{\beta}, \gexp{}{\vc{w}}),
				\quad
			\msk \gets \gexp{}{\alpha \beta}$
				\\
			Return $(\mpk, \msk)$
		}
	&
	\algorithm{
		$\cd{KeyGen}(\vc{d}, \msk):$
		}
		{
			$r \rgets \F_q$
				\\
			$\sk_\vc{d} \gets (\gexp{}{r}, \gexp{}{\alpha \beta + r \vc{w}^\trs \vc{d}})$
				\\
			Return $\sk_\vc{d}$
		}
	\\
	\algorithm{
			$\cd{Enc}(m, \vc{d}, \mpk):$
			}
			{
				$\rho \rgets \F_q$
					\\
				$c_0 \gets \gexp{}{\rho}
					, \quad 
				c_1 \gets \gexp{1}{\rho \vc{w}^\trs \vc{d}}$
					\\
				$c_2 \gets m \cdot e(\gexp{}{\alpha}, \gexp{}{\beta})^\rho$
					\\
				Return $(c_0, c_1, c_2)$
			}
	&
	\algorithm{
		$\cd{Dec}(c, \sk_\vc{d}, \mpk):$
		}
		{
			Parse $c = (c_0, c_1, c_2)$
				\\
			Parse $\sk_\vc{d} = (\sigma_1, \sigma_2)$
				\\
			$m \gets c_2 \cdot e(c_0, \sigma_2) \cdot e(c_1, \sigma_1)^{-1}$
				\\
			Return $m$
		}
\end{pcarray}
\caption{Waters IBE. Identities are of the form $\vc{d} \in \{1\} \times \{0, 1\}^n$}
\label{prot:Waters05:IBE}
\end{figure}

\begin{proposition}
	The scheme in Fig.~\ref{prot:Waters05:IBE} is an $\cd{IND-CPA}$ IBE under the $\cd{BDH}$ assumption
\end{proposition}

\begin{proof}[Proof sketch]
	In order to reduce security to the $\cd{BDH}$ problem, given an instance $h_1 = \gexp{}{\alpha}, h_2 = \gexp{}{\beta}, k = \gexp{}{\gamma}, c^\ast$, we need to build a simulator achieving two orthogonal goals:
	First he has to give secret keys for queried identities and second, he has to embed the problem in the challenge ciphertext.
	
	For the first goal, if we just set $\mpk = (h_1, h_2, \gexp{}{\vc{w}})$ succeeding in producing any key would imply breaking hard problems like $\cd{CDH}$. Our next best option is to sacrifice knowledge on the exponents $\vc{w}$ and $r$, for instance by setting
	\[
		\vc{w} = \vc{v} + \alpha \vc{x},
			\quad
		r = s + \beta y
	\]
	Then, the exponent of $\sk_\vc{d}$ becomes
	\begin{align*}
		\alpha \beta + (s + \beta y) \cdot(\vc{v} + \alpha \vc{x})^\trs \vc{d} 
			& \; =
		\alpha \beta 
			+ 
		s \cdot \vc{w}^\trs \vc{d} 
			+ 
		\beta y \cdot \vc{v}^\trs \vc{d} 
			+ 
		\alpha \beta y \cdot \vc{x}^\trs \vc{d}
	\end{align*}
	In order to cancel out $\alpha \beta$ we need $1 - y \vc{x}^\trs \vc{d} = 0$, which can be obtained by setting $y = -(\vc{x}^\trs \vc{d})^{-1}$ if this inner product is non-zero.
	For clarity, calling $\vc{u} = \gexp{}{\vc{v} + \alpha \vc{x}}$, one would compute the secret key by setting
	\[
		y \gets -\frac{1}{\vc{x}^\trs \vc{d}}
			, \qquad
		\sigma_0 \gets \gexp{}{r} \cdot h_2^{y}
			, \qquad
		\sigma_1 \gets 
			\vc{u}^{s \vc{d}}
				\cdot
			h_1^{y \cdot \vc{v}^\trs \vc{d}},
	\]
	and finally $\sk_\vc{d} \gets (\sigma_0, \sigma_1)$. 
	Next, to simulate the challenge ciphertext we can easily set $c_0 \gets k$ and $c_2 \gets m \cdot c^\ast$.
	The hard part is computing the middle term $c_2$. 
	It's exponent is $\rho \cdot \vc{v}^\trs \vc{d} + \rho \alpha \cdot \vc{x}^\trs \vc{d}$.
	Clearly we can compute the first term but not the second one. Hence, to make this step simulatable we need $\vc{x}^\trs \vc{d} = 0$.
	
	To sum up, calling $\vc{d}_1, \ldots, \vc{d}_Q$ the identity keys queried by the adversary and $\vc{d}_0$ the one used in the challenge, our simulation strategy works if
	\[
	\begin{cases}
		\vc{x}^\trs \vc{d}_i \neq 0 \quad \forall i \in \range{Q}
			\\
		\vc{x}^\trs \vc{d}_0 = 0
	\end{cases}
	.
	\]
	The proof's core is then showing how to chose $\vc{x}$ in such a way that the above condition happens with significant probability.
	This can be done either as shown in the original paper or with standard techniques from programmable hash functions \cite{C:HofKil08}.
\end{proof}