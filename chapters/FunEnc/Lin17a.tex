%!TeX root = ../../ccc_main.tex
%!TeX spellcheck = en_GB

%===============================================================================
% Local Commands
\newcommand{\IPsetup}{\cd{IP.Setup}}
\newcommand{\IPenc}{\cd{IP.Enc}}
\newcommand{\IPkeygen}{\cd{IP.KeyGen}}
\newcommand{\IPdec}{\cd{IP.Dec}}


%===============================================================================

\subsection{Lin Weak Function Hiding Secret Key IPFE}
In \cite{C:Lin17}, Section 6, the first pairing based rate-1 construction\footnote{i.e. such that to encrypt a vector of length $n$ the ciphertext has length $n + o(n)$.} of a function hiding IPFE was proposed.
This improved the previous construction \cite{AC:BisJaiKow15} which expanded encrypted vectors by a factor $2$ and returned upon decryption elements of the form
\[
	\gexp{T}{\theta}
		, \;
	\gexp{T}{\theta \cdot \vc{x}^\trs \vc{y}}
\]
with $\vc{x}$ being the encrypted vector and $\vc{y}$ the vector linked to the secret key.
Although this allows to recover $\vc{x}^\trs \vc{y}$ when this values is known to lie in a polynomially small set, it fails to cover the more general case of computing the generalized pairing
\[
	e(\gexp{1}{\vc{x}}, \gexp{2}{\vc{y}})
		\; = \;
	\gexp{T}{\vc{x}^\trs \vc{y}}.
\]
%
To solve this issue, \cite{C:Lin17} abandoned the dual pairing vector spaces paradigms of Okamoto and Takashima \cite{PAIRING:OkaTak08, AC:OkaTak09}, building instead almost generically upon the IPFE in \cite{PKC:ABDP15}.
The key properties of this scheme are:
\begin{enumerate}
	\item Decryption in \cite{PKC:ABDP15} consist of evaluating an ``inner product'' with ciphertext in $\G^{n+1}$ and secret key in $\F_q^{n+1}$.
	
	\item The decryption algorithm is linear in the vector $\vc{d}$ which describes the function.
	Thus, it can be generalized to vectors of group elements in $\G_1^n$ or $\G_2^n$.
\end{enumerate}
Given them, the core idea to encrypt $\vc{x}$ and generate a secret key for $\vc{y}$ is to apply the IPFE twice: 
$\vc{x}_0$ is first encrypted into $\gexp{1}{\vc{x}_1} \in \G_1^{n+1}$ while from $\vc{y}_0$ a related secret key $\vc{y}_1 \in \F_q^{n+1}$ is computed.
Next, having setted up a new instance of the IPFE for vectors of length $n+1$, in order to hide $\vc{y}_0$, $\vc{y}_1$ is encrypted into $\gexp{2}{\vc{y}_2} \in \G_2^{n+2}$ and $\gexp{1}{\vc{x}_1}$ is used as input in the key generation procedure to get $\gexp{2}{\vc{x}_2} \in \G_1^{n+2}$.
In this way performing a pairing inner product between $\vc{c}$ and $\sk_\vc{y}$ returns
\[
	\gexp{T}{\vc{x}_2^\trs \vc{y}_2}
		\; = \;
	\gexp{T}{\vc{x}_1^\trs \vc{y}_1}
		\; = \;
	\gexp{T}{\vc{x}^\trs \vc{y}}
\]
where all the equality follow as the result of each of these inner products is also the result of the respective decryption procedure.
More formally the scheme is described in Fig.~\ref{prot:Lin17:wfh_IPFE} explicitly.

\begin{figure}[htb]
\centering
\begin{pcarray}{lll}
	\algorithm{
		$\cd{Setup}(1^\secp, n):$
		}
		{
			$\vc{v}_1 \rgets \F_q^n, \; \vc{v}_2 \rgets \F_q^{n+1}$
				\\
			Return $\msk = (\vc{v}, \vc{w})$
 		}
	&
	\algorithm{
		$\cd{KeyGen}(\vc{y}, \msk):$
		}
		{
			Sample $r \rgets \F_q$
				\\
			$\vc{y}_1 \gets (-\vc{y}^\trs \vc{v}_1, \; \vc{y})$
				\\
			$\gexp{2}{\vc{y}_2} \gets \left(\gexp{2}{r}, \; \gexp{2}{\vc{y}_1 + r \vc{v}_2}\right)$
				\\
			Return $\sk = \gexp{2}{\vc{y}_2}$
		}
	\\
	\algorithm{
		$\cd{Enc}(\vc{x}, \msk):$
		}
		{
			Sample $s \rgets \F_q$
				\\
			$\gexp{1}{\vc{x}_1} \gets \left( \gexp{1}{s},\; \gexp{1}{\vc{x} + s \vc{v}_1} \right)$
				\\
			$\gexp{1}{\vc{x}_2} \gets \left( \gexp{1}{-\vc{x}_1^\trs \vc{v}_2},\; \gexp{2}{\vc{x}_1} \right)$
				\\
			Return $\vc{c} = \gexp{1}{\vc{x}_2}$
		}
	&
	\algorithm{
		$\cd{Dec}(\vc{c}, \sk):$
		}
		{
			Return $e(\vc{c}, \sk)$
		}
\end{pcarray}
\caption{Weakly secure function hiding IPFE in \cite{C:Lin17}.}
\label{prot:Lin17:wfh_IPFE}
\end{figure}

\begin{proposition}
	The scheme in Fig.~\ref{prot:Lin17:wfh_IPFE} is a weakly selectively secure function-hiding secret-key IPFE under the SXDH assumption. 
\end{proposition}
\begin{proof}[Proof sketch]
	\EG{TODO: Clarify the definition of selective security in this context (all queries are done together). This weak security notion comes from the usage of ABDP, which is only selectively secure.}
\end{proof}