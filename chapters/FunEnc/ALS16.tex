%!TeX root = ../../ccc_main.tex
%!TeX spellcheck = en_GB

\subsection{Agrawal-Libert-Stehl\'{e} Adaptively Secure IPFE from DDH}

The main limitation of \cite{PKC:ABDP15} is that in the standard model only selective security can be achieved, an issue inherited by primitives based on it.
To address this issue, \cite{C:AgrLibSte16} proposed a variant achieving adaptive security at the cost of $1$ extra group element in ciphertexts and public key, and one extra field element in secret keys.
The system is inspired by the Cramer-Shoup cryptosystem \cite{C:CraSho98} or more generally Hash Proof Systems \cite{EC:CraSho02}.
At a high level the main observations they build upon are
\begin{enumerate}
	\item A \textit{private key} version of \cite{PKC:ABDP15}, where the challenger only encrypts the challenge ciphertext would be adaptively secure.
	In the security proof this is implicitly proven to be true information-theoretically.
	
	\item Using Hash Proof Systems it is possible to setup both a public key and private key instance of \cite{PKC:ABDP15} and at any time produce a ciphertext with the private-key system.
\end{enumerate}

A detailed description is given in Fig.~\ref{prot:AgrLibSth16:IPFE}.
Correctness follows since
\[
	m
		\; = \;
 	\gexp{}{\vc{d}^\trs \left(\vc{x} + A \vc{w} \right)}
 		-
 	\gexp{}{\vc{d}^\trs A \vc{w}}
 		\; = \;
 	\gexp{}{\vc{d}^\trs\vc{x}}.
\]

\begin{figure}[htb]
\centering
\begin{pcarray}{lll}
	\algorithm{
		$\cd{Setup}(1^\secp, n):$
		}
		{
			$A \rgets \F_q^{n, 2}, \; \vc{w} \rgets \F_q^2$
				\\
			$\mpk \gets \left( \gexp{}{\vc{w}}, \gexp{}{A\vc{w}} \right), \, \msk \gets \left(A, \vc{w}\right)$
				\\
			Return $(\mpk, \msk)$
		}
	&
	\algorithm{
		$\cd{KeyGen}(\vc{d}, \msk):$
		}
		{
			$\sk_\vc{d} \gets \vc{d}^\trs A$
				\\
			Return $\sk_\vc{d}$
		}
	\\
	\algorithm{
		$\cd{Enc}(\vc{x}, \mpk):$
		}
		{
			Sample $r \rgets \F_q$
				\\
			$c \gets \left(\gexp{}{r \vc{w}}, \gexp{}{\vc{x} + r A \vc{w}} \right)$
				\\
			Return $c$
		}
	&
	\algorithm{
		$\cd{Dec}(c, \sk_\vc{d}, \mpk):$
		}
		{
			Parse $c = (\vc{c}_0, \vc{c}_1) \in \G^2 \times \G^n$
				\\
			$m \gets \vc{d}^\trs \vc{c}_1 - \sk_\vc{d}^\trs \cdot \vc{c}_0$
				\\
			Return $m$
		}
\end{pcarray}
\caption{Agrawal-Libert-Sthel\'{e} Adaptively Secure IPFE from DDH.}
\label{prot:AgrLibSth16:IPFE}
\end{figure}

\begin{proposition}
	The scheme in Fig.~\ref{prot:AgrLibSth16:IPFE} is an Adaptively Secure Functional Encryption for Inner Production is $\ddh$ is hard in $\G$.
\end{proposition}
\begin{proof}
	First, we define the following hybrid games, with $b \sim U(\{0, 1\})$ being the bit used in the indistinguishability game \EG{ref}.
	\begin{itemize}
		\item $\cd{G}_0$: The real game \EG{ref?}.
		
		\item $\cd{G}_1$: As $\cd{G}_0$ but when the adversary submit the challenge messages $\vc{x}_0, \vc{x}_1$, the ciphertext is computed sampling $\vc{z} \rgets \F_q^2$ and setting
		\[
			c 
				\; = \; 
			\left( \gexp{}{\vc{z}}, \; \gexp{}{\vc{x}_b + A \vc{z}} \right).
		\]
	\end{itemize}
	We will first prove that $\cd{G}_0$ is indistinguishable from $\cd{G}_1$ using $\ddh$ and then that any adversary in $\cd{G}_1$ has negligible advantage $q^{-1}$.\\
	
	For the first part, let $\mathcal{A}$ be a distinguisher between $\cd{G}_0$ and $\cd{G}_1$. 
	We define $\mathcal{B}$ which breaks $\ddh$ using $\mathcal{A}$.
	On input $\gexp{}{\alpha}, \gexp{}{\beta}, \gexp{}{\gamma}$ it samples a random $\rho \rgets \F_q$, $A \rgets \F_q^{n, 2}$ and set
	\[
		\gexp{}{\vc{w}}
			\; = \;
		\gexp{}{(\rho, \alpha)}
			, \qquad
		\gexp{}{\vc{z}}
			\; = \;
		\gexp{}{(\rho\beta, \gamma)}.
	\]
	Thus it sets the master public key as $\gexp{}{\vc{z}}, \, \gexp{}{A \vc{z}}$ applying $A$ in the exponent.
	Key generation queries for $\vc{d}$ are answered with $\vc{d}^\trs A$ as prescribed while the challenge ciphertext is computed setting
	\[
		c 
			\; = \;
		\left( \gexp{}{\vc{w}}, \; \gexp{}{\vc{x}_b + A \vc{w}} \right)
	\]
	again applying $A$ in the exponent to get $\gexp{}{A \vc{w}}$.
	In $\ddh[1]$, $\mathcal{B}$ correctly simulates game $\cd{G}_0$ as $\gamma = \alpha \beta$ implies that $\vc{z} = \beta \vc{w}$ and in particular the challenge ciphertext is correctly distributed.
	Conversely in $\ddh[0]$, $\vc{w} \sim U(\F_q^2)$ and independently with other messages sent in the challenge, as in $\cd{G}_1$.
	Hence $\adv{\mathcal{B}} = \adv{\mathcal{A}}$.\\
	
	For the second part, up to a negligible probability $q^{-1}$, $\vc{z}$ and $\vc{w}$ are linearly independent.
	Conditioning on this event we show that the view of a given adversary $\mathcal{A}$ follows the same distribution when $b = 0$ or $b = 1$.
	Calling $\vc{y}_1, \ldots, \vc{y}_t$ all the key generation queries requested throughout the execution, and let $Y = (\vc{y}_1, \ldots, \vc{y}_t)$.
	Notice that $\vc{y}_i^\trs \vc{x}_0 = \vc{y}_i^\trs \vc{x}_1$ implies that $\vc{y}_i$ is orthogonal to $\vc{x}_1 - \vc{x}_0$ and in particular $Y^\trs(\vc{x}_1 - \vc{x}_0) = \vc{0}$.
	With this notation adversary's view is a function of 
	\[
		D_b \; = \; (A\vc{w}, \; Y^\trs A, \; \vc{x}_b + A \vc{z}).
	\]
	Since $\vc{w}, \vc{z}$ are linearly independent, there exists a matrix $B \in \F_q^{n, 2}$ such that $B \vc{w} = \vc{0}$ and $B \vc{z} = \vc{x}_1 - \vc{x}_0$.
	Given this matrix we conclude that $D_0$ and $D_1$ follows the same distribution since $A + B \sim U(\F_q^{n, 2})$ and, applying the transformation $A \mapsto A + B$ to $D_0$
	\[
		(A + B) \vc{w} 
			\, = \, 
		A \vc{w},
			\qquad
		Y^\trs(A + B)
			\, = \,
		Y^\trs A,
			\qquad
		\vc{x}_0 + (A + B) \vc{z}
			\, = \,
		\vc{x}_1 + A \vc{z}
	\]
	where the second equation follows as $Y^\trs B \in \F_q^{t, 2}$ is the zero matrix because it maps $\vc{z}, \vc{w}$ (a base of $\F_q^2$) to the zero vector.
	Thus $\adv{A} \leq q^{-1}$ in $\cd{G}_1$, concluding the proof.
\end{proof}