\subsection{PVW Lossy Encryption from DDH}
\EG{NOTE: we use the group in additive notation}
Although the formal definition of \textit{lossy encryption} was first given in \cite{AC:HLOV11}, the first construction and motivations dates back to \cite{C:PeiVaiWat08}.
In that paper the primitive was presented as \textit{dual-mode} encryption where the public key can be generated in two indistinguishable ways.
In \textit{injective} mode, each ciphertext produced with it preserve the plaintext and allows for correct decryption.
Conversely in \textit{messy} or \textit{lossy} mode, encryption retain almost no information on the message.

In the following we present a slightly modified version of their scheme based on DDH.
First, it generalizes ElGamal to the group $\G^2$ for regular encryption, with public key in injective mode being $\vc{g}, \vc{h}$ with $\vc{h} = x \cdot \vc{g}$. Concretely, for message $m \in \G$ and randomness $\vc{r} \in \F_q^2$
\[
	\vc{c} = \left( \vc{r}^\trs \vc{g}, \; \vc{r}^\trs \vc{h} + m \right)
\]
Conversely lossy keys are generated choosing $\vc{g}, \vc{h}$ both random.
The idea is that in this way the encryption of any message will be uniformly distributed over the ciphertext space even if the discrete logarithms of $\vc{g}$ and $\vc{h}$ are known.
The full construction presented in Figure~\ref{prot:PVW08}

\begin{figure}[htb]
\centering
\begin{pcarray}{ll}
	\algorithm{
		$\cd{Setup}(1^\secp):$
		}
		{
			$\sk = x \rgets \F_q$ the secret key
				\\
			$\vc{g} \rgets \G^2, \; \vc{h} \gets x \cdot \vc{g}$
				\\
			Return $(\pk, \sk)$ with $\pk = (\vc{g}, \vc{h})$	
		}
	&
	\algorithm{
		$\cd{LossySetup}(1^\secp):$
		}
		{
			$\vc{g}, \vc{h} \rgets \G^2$
				\\
			Return $\pk = (\vc{g}, \vc{h})$
		}
	\\
	\algorithm{
		$\cd{Enc}(m, \pk)$
		}
		{
			Parse $\pk = (\vc{g}, \vc{h})$
				\\
			Sample $\vc{r} \rgets \F_q^2$
				\\
			$c_1 \gets \vc{r}^\trs \vc{g}, \; c_2 \gets \vc{r}^\trs \vc{h} + m$
				\\
			Return $\vc{c} = (c_1, c_2)$
		}
	&
	\algorithm{
		$\cd{Dec}(c, \sk):$
		}
		{
			Parse $c = (c_1, c_2) \in \G^2$
				\\
			Return $m \gets c_1 - x \cdot c_2$
		}
\end{pcarray}
\label{prot:PVW08}
\caption{\cite{C:PeiVaiWat08} encryption for HHS, with message space ${0, 1}^\mu$}
\end{figure}

\begin{proposition}
	\label{prop:GVW08}
	The scheme described in Figure~\ref{prot:PVW08} is an IND-CPA secure lossy encryption scheme if DDH is hard.
\end{proposition}
\begin{proof}
	\EG{TODO: IND-CPA is like el-gamal. Key indistinguishability is plain DDH. Lossiness comes for the fact that for any $m$ the map $\vc{r} \mapsto \cd{Enc}(\pk, m; \vc{r})$ is surjective}
\end{proof}
