\subsection{Sahai-Waters from iO}

An important question on the path to render iO the "central hub" of modern cryptography is under which assumptions it implies fundamental primitives such as public-key encryption.
The first answer was provided by \cite{STOC:SahWat14}.
Their idea is to start with a simple secret key encryption from PRF $f$, which encrypts $m$ as $(r, f_k(r) \oplus m)$ with $r$ a random nonce.
Then this scheme is compiled into a public-key one.
The ingredients required for such transformation are
\begin{itemize}
	\item A puncturable PRF ($\cd{PRF.Gen}, \cd{PRF.Puncture}, \cd{PRF.Eval}$) \EG{cite}
	\item A pseudo-random generator $g : \{0,1\}^\secp \to \{0,1\}^{2\secp}$
	\item Indistinguishability obfuscation $\io$
\end{itemize}
Given the above $f_k(\cdot)$ is replaced with $\cd{PRF.Eval}(k, \cdot)$.
The randomness $r$ instead is sampled through $g(s)$ with short seed $s$, which creates a (relatively sparse) subset of "obtainable ciphertexts" in the space of all "valid" ones.
With these modifications, the public key of the resulting scheme is set as the obfuscation of the circuit evaluating the secret-key encryption.
\[
	C_k(m, s) = \left(
		g(s),
			\,
		\cd{PRF.Eval}(k, g(s)) \oplus m
	\right).
\]
The full scheme is described in figure \ref{prot:SahWat14:pke_from_io}.

\begin{figure}[htb]
	\centering
	\begin{pchstack}[center, space=12pt]
	\algorithm{
		$\cd{Gen}(1^\secp):$
		}
		{
			Sample $k \rgets \cd{PRF.Gen}(1^\secp)$
				\\
			Obfuscate $\tilde{C} \gets \io(C_k)$
				\\
			$\pk \gets \tilde{C}, \, \sk \gets k$
				\\
			\textbf{Return} $(\pk, \sk)$
		}
	\algorithm{
		$\cd{Enc}(\pk, m):$
		}
		{
			Sample a PRG seed $s$
				\\
			$c \gets \tilde{C}(m, s)$
				\\
			\textbf{Return} $c$
		}
	\end{pchstack}
	\begin{pchstack}[center, space=12pt]
	\algorithm{
		$\cd{Dec}(\sk, c):$
		}
		{
			Parse $c = (c_1, c_2)$
				\\
			\textbf{Return} $\cd{PRF.Eval}(k, c_1) \oplus c_2$
		}
	\end{pchstack}
	\label{prot:SahWat14:pke_from_io}
	\caption{Circuit evaluating the modified encryption scheme.}
\end{figure}

\begin{proposition}
	\label{prop:SahWat98}
	If $(\cd{PRF.Gen}, \cd{PRF.Puncture}, \cd{PRF.Eval})$ is a secure puncturable PRF, $g$ is a PRG and $\io$ is an indistinguishability obfuscator for P/poly, then the scheme in Figure~\ref{prot:SahWat14:pke_from_io} is IND-CPA secure.
\end{proposition}
\begin{proof}
	The proof follows from a sequence of four hybrid games, where we call $m_0, m_1$ the challenge messages chosen by the adversary and $b$ the bit sampled by the challenger.
	\begin{itemize}
		\item[$\cd{H}_0$:] The real IND-CPA game with $c = (g(s), \cd{PRF.Eval}(k, g(s)) \oplus m_b)$.
		
		\item[$\cd{H}_1$:] As $\cd{H}_0$ but $r^\ast$ is initially sampled and the challenge ciphertext is set as $c = (r^\ast, \cd{PRF.Eval}(k, r^\ast) \oplus m_b)$.
		
		\item[$\cd{H}_2$:] As $\cd{H}_1$ but $k^\ast \gets \cd{PRF.Puncture}(k, \{r^\ast\})$ and $\tilde{C} = \io(C_{k^\ast})$.
		
		\item[$\cd{H}_3$:] As $\cd{H}_2$ but $c = (r^\ast, z^\ast)$ with $z^\ast$ uniformly random.
	\end{itemize}
	At a high level $\cd{H}_0 \approx \cd{H}_1$ based on the PRG security replacing $g(s)$ with $r^\ast$.
	$\cd{H}_1 \approx \cd{H}_2$ uses the fact that, since $r^\ast$ is sampled uniformly over $\{0,1\}^{2\secp}$, the probability that $r^\ast \in \Im g$ is $2^{-\secp}$.
	Thus, conditioning on $r^\ast \notin \Im g$ we have that $C_k$ and $C_{k^\ast}$ are functionally equivalent.
	In particular the hybrid are indistinguishable due to the security of the obfuscator.
	Finally, from the pseudo-randomness of the PRF, as $k^\ast$ is punctured in $r^\ast$, the value $\cd{PRF.Eval}(k^\ast, r^\ast)$ is computationally indistinguishable from true randomness form the adversary's view.
	In particular so is $\cd{PRF.Eval}(k^\ast, r^\ast) \oplus m_b$.
\end{proof}



